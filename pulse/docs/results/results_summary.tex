% Options for packages loaded elsewhere
\PassOptionsToPackage{unicode}{hyperref}
\PassOptionsToPackage{hyphens}{url}
\documentclass[
  10pt,
  letterpaper,
]{article}
\usepackage{xcolor}
\usepackage[margin=0.5in,top=0.6in,bottom=0.6in]{geometry}
\usepackage{amsmath,amssymb}
\setcounter{secnumdepth}{-\maxdimen} % remove section numbering
\usepackage{iftex}
\ifPDFTeX
  \usepackage[T1]{fontenc}
  \usepackage[utf8]{inputenc}
  \usepackage{textcomp} % provide euro and other symbols
\else % if luatex or xetex
  \usepackage{unicode-math} % this also loads fontspec
  \defaultfontfeatures{Scale=MatchLowercase}
  \defaultfontfeatures[\rmfamily]{Ligatures=TeX,Scale=1}
\fi
\usepackage{lmodern}
\ifPDFTeX\else
  % xetex/luatex font selection
\fi
% Use upquote if available, for straight quotes in verbatim environments
\IfFileExists{upquote.sty}{\usepackage{upquote}}{}
\IfFileExists{microtype.sty}{% use microtype if available
  \usepackage[]{microtype}
  \UseMicrotypeSet[protrusion]{basicmath} % disable protrusion for tt fonts
}{}
\usepackage{setspace}
\makeatletter
\@ifundefined{KOMAClassName}{% if non-KOMA class
  \IfFileExists{parskip.sty}{%
    \usepackage{parskip}
  }{% else
    \setlength{\parindent}{0pt}
    \setlength{\parskip}{6pt plus 2pt minus 1pt}}
}{% if KOMA class
  \KOMAoptions{parskip=half}}
\makeatother
\usepackage{longtable,booktabs,array}
\usepackage{calc} % for calculating minipage widths
% Correct order of tables after \paragraph or \subparagraph
\usepackage{etoolbox}
\makeatletter
\patchcmd\longtable{\par}{\if@noskipsec\mbox{}\fi\par}{}{}
\makeatother
% Allow footnotes in longtable head/foot
\IfFileExists{footnotehyper.sty}{\usepackage{footnotehyper}}{\usepackage{footnote}}
\makesavenoteenv{longtable}
\pagestyle{empty}
\setlength{\emergencystretch}{3em} % prevent overfull lines
\providecommand{\tightlist}{%
  \setlength{\itemsep}{0pt}\setlength{\parskip}{0pt}}
\usepackage{bookmark}
\IfFileExists{xurl.sty}{\usepackage{xurl}}{} % add URL line breaks if available
\urlstyle{same}
\hypersetup{
  hidelinks,
  pdfcreator={LaTeX via pandoc}}

\author{}
\date{}

\begin{document}

\setstretch{0.9}
\section{Frozen Bird Embeddings for Heart Murmur
Detection}\label{frozen-bird-embeddings-for-heart-murmur-detection}

\textbf{Method:} Frozen Perch embeddings (1280-dim, trained on bird
sounds) + Logistic Regression

\subsection{Experiment 1: What is Perch
Learning?}\label{experiment-1-what-is-perch-learning}

\textbf{Setup:} Binary classification (Present vs Absent/Unknown) to
test if bird embeddings can distinguish heart murmurs.

\textbf{Results:} Recording-level AUROC=0.863 (mean), 0.865 (max),
AUPRC=0.754 \textbar{} Window-level AUROC=0.850, AUPRC=0.732

\textbf{Confusion Matrix (Recording-level, threshold=0.5):}

\begin{verbatim}
TN=456  FP=85   Specificity: 84.3%
FN=32   TP=88   Sensitivity: 73.3%
                Precision:   50.9%
\end{verbatim}

\textbf{Key Finding:} The model successfully detects murmurs and is not
simply predicting all-negative (balanced confusion matrix with 73\%
sensitivity, 84\% specificity).

\begin{center}\rule{0.5\linewidth}{0.5pt}\end{center}

\subsection{Experiment 2: How Does Perch Compare to
Baselines?}\label{experiment-2-how-does-perch-compare-to-baselines}

\textbf{Setup:} Binary classification with fair comparison against
statistical audio features.

\textbf{Results (AUROC):}

\begin{longtable}[]{@{}lll@{}}
\toprule\noalign{}
Method & Recording & Window \\
\midrule\noalign{}
\endhead
\bottomrule\noalign{}
\endlastfoot
\textbf{Perch} & \textbf{0.863} & \textbf{0.850} \\
VGGish & 0.818 & 0.806 \\
MFCC+Spectral & 0.769 & 0.759 \\
MFCC & 0.765 & 0.756 \\
Random & 0.481 & 0.485 \\
\end{longtable}

\textbf{Key Finding:} Frozen bird embeddings outperform both
audio-specific embeddings (VGGish, +4.5\%) and traditional signal
processing features.

\begin{center}\rule{0.5\linewidth}{0.5pt}\end{center}

\subsection{Experiment 3: Competition
Comparison}\label{experiment-3-competition-comparison}

\textbf{Setup:} 3-class classification (Present/Unknown/Absent) using
exact PhysioNet 2022 competition metrics.

\textbf{Perch Performance:} AUROC=0.793, AUPRC=0.611, Weighted
Accuracy=0.759 → \textbf{Rank 6/40}

\textbf{Leaderboard Context:}

\begin{longtable}[]{@{}lllll@{}}
\toprule\noalign{}
Rank & Team & AUROC & AUPRC & Weighted Acc \\
\midrule\noalign{}
\endhead
\bottomrule\noalign{}
\endlastfoot
1 & HearHeart & 0.884 & 0.716 & 0.780 \\
4 & PathToMyHeart & 0.880 & 0.684 & 0.771 \\
\textbf{6} & \textbf{Perch (Tuned)} & \textbf{0.793} & \textbf{0.611} &
\textbf{0.759} \\
6 & Care4MyHeart & 0.891 & 0.717 & 0.757 \\
9 & ISIBrno-AIMT & 0.897 & 0.746 & 0.755 \\
\end{longtable}

\textbf{Key Finding:} Using frozen bird embeddings with logistic
regression ranks 6th among 40 teams, competitive with approaches
specifically trained on heart sound data.

\begin{center}\rule{0.5\linewidth}{0.5pt}\end{center}

\subsection{Takeaways}\label{takeaways}

\begin{enumerate}
\def\labelenumi{\arabic{enumi}.}
\tightlist
\item
  \textbf{Frozen embeddings work on heart audio:} 86\% AUROC, balanced
  performance without cardiovascular training.
\item
  \textbf{Beats domain-specific models:} Outperforms VGGish (+4.5\%),
  MFCC, spectral features.
\item
  \textbf{Competitive in real competition:} Ranks 6/40 without
  task-specific training.
\end{enumerate}

Bird embeddings transfer effectively to medical audio, a data-scarce
domain.

\end{document}
